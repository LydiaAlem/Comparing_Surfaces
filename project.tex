\documentclass[english]{article}
\usepackage{graphicx}
\usepackage{subfigure}
\usepackage[T1]{fontenc}
\usepackage[latin9]{inputenc}
\usepackage{babel}
\usepackage{amsmath}
\usepackage{amsthm}
\usepackage{amssymb}
\usepackage{amsfonts}
\usepackage{enumerate}
\usepackage{fancyhdr}
\usepackage{float}
\usepackage{url}
\usepackage{hyperref}
\usepackage{natbib}
\usepackage{pgfplots}
\pgfplotsset{compat=1.17}
\usepackage{subcaption}
\newcommand{\LaTeXs}{\LaTeX{\/} }
\newcommand{\matlabs}{{\sc Matlab{\/} }}
\newcommand{\matlab}{{\sc Matlab}}
\begin{document}

\title{MATH 2263: Project \#1: Comparing Surfaces}

\author{
Lydia Alem
}
% \date{xx/xx/xx} %uncomment this to place a specific date or leave it blank to erase the date 

\maketitle

\section{Part 1: Traces}\label{introduction}
% - first create a contour map of each surface the paraboloid and the cone
% Answer the questions:
    % - What are the similarities between these contour maps
    The cone and paraboloid are two mathematical surfaces with distinct shapes that, at first glance, may appear to be similar. A cursory examination of both Figure 1 and Figure 2 reveals that both shapes exhibit an almost circular-like distribution. However, upon delving deeper into this project, I will demonstrate how significantly different these shapes are. 
    
    \begin{figure}[H]
    	\begin{centering}
      		\includegraphics[width=5in]{nig.jpg}
      		% \caption{Contour map of both {$Z = \sqrt{x^2 + y^2}$} and $Z = x^2 + y^2$ 
        % }
    	\end{centering}
    \end{figure}

    
    
    % - Q: How are they different

    Upon observing the contour maps of the cone and paraboloid, a notable disparity in their growth patterns is immediately discernible. Specifically, the paraboloid exhibits a rate of growth that could be described as approaching an exponential function, evidenced by the discrepancy in growth between $z = 4$ and $z = 3$ when compared to the difference in growth between $z = 2$ and $z = 3$. 
    \\
    \\
    In contrast, the growth of the cone appears to be more uniform across its dimensions.
    This observation could be attributed to the paraboloid's underlying quadratic equation, which imparts a degree of curvature and accentuates the rate of growth in the $z$ direction. Such distinctions are critical to understanding the properties and applications of these shapes in various fields, including physics, engineering, and mathematics.
     
\section{Part 2: Directional Derivatives}\label{methods}
% - Use the directional derivatives  to find the direction of greatest increase at this point on each surface, and also find the magnitude of the directional derivative in that direction. 

I began by calculating the directional derivatives and magnitude of both shapes:


\subsection{Step 1}

Find the Gradient Vector and Magnitude for $z = x^2 + y^2 $:

$$\fbox{\textbf{Recall the Gradient Vector Formula: $\nabla f = \bigl \langle f_{x}, f_{y}, f_{z} \bigr \rangle = \frac{\partial f}{\partial x} \bold{i} + \frac{\partial f}{\partial y} \bold{j} + \frac{\partial f}{\partial z} \bold{k}$}}$$

\begin{align*}
    \frac{\partial f}{\partial x} &= 2x, \quad
    \frac{\partial f}{\partial y} = 2y, \quad
    \frac{\partial f}{\partial z} = -1 
    \longrightarrow \nabla f = \bigl \langle 2x, 2y, -1 \bigr \rangle
\end{align*}



Plugging in the point $(1,0,1)$, we have:

 $$ \nabla f(1,0,1) = \bigl \langle 2, 0, -1 \bigr \rangle $$

\begin{align*}
    \text{\textbf{Gradient Vector: }} \bigl \langle 2, 0, -1 \bigr \rangle
\end{align*}

$$\left\lVert \bigl \langle 2, 0, -1 \bigr\rangle \right\rVert = \sqrt{2^2 + 0^2 + (-1)^2} = \sqrt{5} $$

\begin{align*}
    \text{\textbf{Magnitude: }} = \sqrt{5}
\end{align*}


\subsection{Step 2}

Find the Gradient Vector and Magnitude for $z =\sqrt{x^2 + y^2 }$:

$$\fbox{\textbf{Recall the Gradient Vector Formula: $\nabla f = \bigl \langle f_{x}, f_{y}, f_{z} \bigr \rangle = \frac{\partial f}{\partial x} \bold{i} + \frac{\partial f}{\partial y} \bold{j} + \frac{\partial f}{\partial z} \bold{k}$}}$$

\begin{align*}
    \frac{\partial f}{\partial x} &= \frac{x}{\sqrt{x^2 + y^2}}, \quad
    \frac{\partial f}{\partial y} &= \frac{y}{\sqrt{x^2 + y^2}}, \quad
    \frac{\partial f}{\partial z} = -1 
    \longrightarrow \nabla f = \bigl \langle \frac{x}{\sqrt{x^2 + y^2}}, \frac{y}{\sqrt{x^2 + y^2}}, -1 \bigr \rangle
\end{align*}



Plugging in the point $(1,0,1)$, we have:

$$ \nabla f(1,0,1) = \bigl \langle 1, 0, -1 \bigr \rangle $$

\begin{align*}
    \text{\textbf{Gradient Vector: }} \bigl \langle 1, 0, -1 \bigr \rangle
\end{align*}

$$\left\lVert \bigl \langle 1, 0, -1 \bigr\rangle \right\rVert = \sqrt{1^2 + 0^2 + (-1)^2} = \sqrt{2} $$

\begin{align*}
    \text{\textbf{Magnitude: }} = \sqrt{2}
\end{align*}



% Continue with as many steps as necessary

\subsection{Final Answer}

Upon analyzing the rate of change of the surface areas of a paraboloid and a cone, it is evident that the paraboloid has the highest rate of increase, with a value of $\sqrt{5}$, while the cone has a rate of increase of $\sqrt{2}$. The computation of the greatest increase highlights the substantial difference between the two shapes, revealing that the paraboloid increases at a much faster rate than the cone.
\\

Furthermore, observing the contour maps for both shapes provides a graphical representation of their rate of change. From the contour map, it is clear that the paraboloid has a much larger increase as compared to the cone, while the latter seems to increase at a relatively constant rate.

\section{Part 3: Tangent Places}
\begin{center}
    \fbox{
        \begin{minipage}{0.7\textwidth}
            \textbf{Suppose that $f$ has a continuous partial derivative. An equation of the tangent plane to the surface. An equation of the tangent plane to the surface $z = f(x,y)$ at the point $P(x_0, y_0, z_0)$ is given by:}\
            \begin{equation*}
                z - z_0 = \frac{\partial f}{\partial x}\bigg|{(x_0,y_0)}(x-x_0)\bigg| + \frac{\partial f}{\partial y}\bigg|{(x_0,y_0)}(y-y_0)\bigg|
            \end{equation*}
    \end{minipage}
    }
\end{center}

\subsection{Step 1}

Find the Tangent Planes on $P(0,0,0)$ for $z = x^2 + y^2$

\begin{align*}
    \frac{\partial f}{\partial x} = 2x, \quad
    \frac{\partial f}{\partial y} = 2y, \quad 
\end{align*}
\noindent Plugging in $x=0$ and $y=0$, we get:
$$ z - 0 = 2x (0,0)(x-0) + 2y (0,0)(y-0) $$
\noindent so the equation of the tangent plane is:
$$ \textbf{z = 0}$$
\qed

\subsection{Step 2}

Find the Tangent Planes on $P(0,0,0)$ for $Z = \sqrt{x^2 + y^2}$

\begin{align*}
    \frac{\partial f}{\partial x} = \frac{x}{\sqrt{x^2 + y^2}}, \
    \frac{\partial f}{\partial y} = \frac{y}{\sqrt{x^2 + y^2}},
\end{align*}

\noindent Plugging in $x=0$ and $y=0$, we get:

\begin{align*}
    z - 0 &= \frac{x}{\sqrt{x^2 + y^2}}(0,0)(x-0) + \frac{y}{\sqrt{x^2 + y^2}} (0,0)(y-0)\
\end{align*}

\noindent so the equation of the tangent plane is:

\begin{align*}
    \mathbf{z = 0}.
\end{align*}
\qed

% Continue with as many steps as necessary
\subsection{Step 3}
Plotting the surface and the tangent

    \begin{figure}[H]
    \centering
    \includegraphics[width=4in]{four.jpg}
    \caption*{3D image of both $Z = \sqrt{x^2 + y^2}$ and $Z = x^2 + y^2$}
\end{figure}



To provide a comprehensive understanding of the cone and paraboloid shapes, I decided to utilize the powerful visualization tool, GeoGebra 3D. This tool allows for an interactive 3D representation of the shapes, which provides a better insight into their structure and properties. 
\\

\subsection{Final Answer}
Therefore, in our effort to derive the equation of the tangent plane to the surface $z=f(x,y)$ at the point $P(0,0,0)$, we have obtained $z=0$ as the equation of the tangent plane. This result indicates that the tangent line at the origin lies on the x-y plane. This finding can also be clearly observed in the accompanying graphs (figure 1 \& 2), where the point $P(0,0,0)$ is seen to be untouched by the surface.


% The following code snippets were used to create the contour maps:

% \begin{tikzpicture}
%     \begin{axis}[view={0}{90},xlabel=$x$,ylabel=$y$,colormap/viridis]
%         \addplot3[contour gnuplot={levels={0,1,2,3,4},labels={false}},thick] 
%             {x^2 + y^2};
%         \addlegendentry{Function}
%         \addplot[colorbar, point meta min=0, point meta max=5] 
%             coordinates{(4,0)};
%         \addlegendentry{Value}
%     \end{axis}
% \end{tikzpicture}

% \begin{figure}[h]
%     \centering
%     \begin{tikzpicture}
%         \begin{axis}[view={0}{90},xlabel=$x$,ylabel=$y$,colormap/viridis]
%             \addplot3[contour gnuplot={levels={0,1,2,3,4},labels={false}},thick] 
%                 {sqrt(x^2 + y^2)};
%             \addlegendentry{Function}
%             \addplot[colorbar, point meta min=0, point meta max=5] 
%                 coordinates{(4,0)};
%             \addlegendentry{Value}
%             \addplot[mark=*, mark size=3, only marks, color=red] 
%                 coordinates{(1,0)} node [above left] {$(1,0)$};
%         \end{axis}
%     \end{tikzpicture}
%     \caption{Contour plot of a $z = \sqrt{x^2 + y^2}$.}
%     \label{fig:contour-plot}
% \end{figure}

% \begin{figure}[h]
%     \centering
%     \begin{tikzpicture}
%         \begin{axis}[view={0}{90},xlabel=$x$,ylabel=$y$,colormap/viridis]
%             \addplot3[contour gnuplot={levels={0,1,2,3,4},labels={false}},thick] 
%                 {x^2 + y^2};
%             \addlegendentry{Function}
%             \addplot[colorbar, point meta min=0, point meta max=5] 
%                 coordinates{(4,0)};
%             \addlegendentry{Value}
%             \addplot[mark=*, mark size=3, only marks, color=red] 
%                 coordinates{(1,0)} node [above left] {$(1,0)$};
%         \end{axis}
%     \end{tikzpicture}
%     \caption{Contour plot of a $z = x^2 + y^2$.}
%     \label{fig:contour-plot}
% \end{figure}
\end{document}

